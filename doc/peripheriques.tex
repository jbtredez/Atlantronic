\chapter{Électronique}

\section{Introduction}

L'électronique se compose principalement d'une carte à 3 étages :
\begin{itemize}
	\item étage alimentation fait maison
	\item étage logique fait maison
	\item étage CPU achetée (carte disco basée sur un stm32f427)
\end{itemize}

\section{Caractéristiques techniques}
\subsection{Carte alimentation}
Les caractéristiques de la carte sont :
\begin{itemize}
	\item alimentation de 15V à 40V (noté VBAT).
		\begin{itemize}
			\item[•] alimentation correcte de la logique à partir de 8V
			\item[•] cramage entrée analogique de mesure tension batterie à partir de 43V et de l'alimentation à partir de 45V.
		\end{itemize}
	\item ARU qui coupe la puissance
	\item alimentation logique 5.03V / 3A
		\begin{itemize}
			\item[•] jusqu'a 3.5A si VBAT > 25V
		\end{itemize}
	\item alimentation puissance 5.03V / 3A (peut être coupée par le logiciel)
		\begin{itemize}
			\item[•] jusqu'a 3.5A si VBAT > 25V
		\end{itemize}
	\item 3 alimentation puissance 11.88V / 3A (peut être coupée par le logiciel)
		\begin{itemize}
			\item[•] jusqu'a 3.4A si VBAT > 25V
		\end{itemize}
	\item fusible 5V logique
	\item fusible puissance (alim 5V puissance et 3 alim 11.88V puissance)
	\item 4 PWM sur VBAT avec retour intensité, 3A par sortie (6A en pic sur 200ms)
		\begin{itemize}
			\item[•] retour intensité avec un facteur théorique de 0.377 V/A
		\end{itemize}
\end{itemize}

\subsection{Carte logique + carte CPU}
Les caractéristiques de la carte sont :
\begin{itemize}
	\item 1 CAN alimenté en 5V logique
		\begin{itemize}
			\item[•] 2 moteurs de propulsion
		\end{itemize}
	\item 2 liens RS232 1Mb/s (USART1 et USART3) avec alimentation 5V de puissance
		\begin{itemize}
			\item[•] 2 hokuyos à 750kb/s
		\end{itemize}
	\item 2 bus séries UART TTL half duplex (UART5) et RS485 20Mb/s (UART4) répartis sur 3 connecteurs (alimentations puissance 11.88V séparées)
		\begin{itemize}
			\item[•] bus ax12 et rx24f 1Mb/s
		\end{itemize}
	\item 2 liens série TTL (USART2 et USART6) avec alimentation 3V logique
		\begin{itemize}
			\item[•] USART2 : module xbee
			\item[•] USART6 : non utilisé
		\end{itemize}
	\item 2 bus SPI avec alimentation 3V
		\begin{itemize}
			\item[•] bus 1 (SPI5)
				\begin{itemize}
					\item gyromètre 3 axes (branché sur la carte CPU)
					\item lcd (branché sur la carte CPU)
					\item 1 lien SPI libre
				\end{itemize}
			\item[•] bus 2 (SPI6)
				\begin{itemize}
					\item 2 liens SPI libres
				\end{itemize}
		\end{itemize}
	\item micro usb OTG 12Mb (carte CPU)
		\begin{itemize}
			\item uniquement mode device programmé pour le moment
		\end{itemize}
	\item mini usb device 12Mb (carte logique)
	\item mini usb : jtag (stlink v2) embarqué pour programmation/débug de la carte (carte logique)
	\item 3 entrées encodeurs analogiques (signaux A/B) avec alimentation logique 5V
	\item 1 io de contrôle des alimentations de puissance
	\item 1 entrée analogique de mesure de la tension batterie (facteur théorique 1/13 )
	\item 1 bouton go déporté
	\item 12 entrées réparties sur 6 connecteurs avec alimentation 5V logique et 11.88V puissance
	\item 4 AN libres
\end{itemize}

\clearpage
